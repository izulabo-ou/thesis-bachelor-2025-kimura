\documentclass[titlepage,11pt]{jarticle}
\usepackage{ics-thesis}

\usepackage[dvipdfmx]{graphicx}
\usepackage{latexsym}
\usepackage{comment}
\usepackage{here}        % [H] placement
\usepackage{enumerate}
\usepackage{bm}
\usepackage{amsmath,amssymb,amsfonts,amsthm}
\usepackage{algorithm}
\usepackage{algpseudocode}

% Hyperref (dvipdfmx + Japanese)
\usepackage[dvipdfmx, bookmarkstype=toc, colorlinks=false,
  pdfborder={0 0 0}, bookmarks=true, bookmarksnumbered=true]{hyperref}
\usepackage{pxjahyper}

\usepackage[normalem]{ulem}
\usepackage{color}
\newcommand{\red}[1]{\textcolor{red}{#1}}
\newcommand{\cancel}[1]{\sout{\textcolor{blue}{#1}}}
\newcommand{\tab}{\hspace*{3mm}}
\newcommand{\abs}[1]{\lvert #1 \rvert}

% =========================================================
% 以下のコマンドは説明用にあるだけです。必要に応じて削除してください。
% =========================================================
\newcommand{\chapterintro}[1]{%
  \noindent\textbf{この章でやること:} #1\par\medskip
}

\newcommand{\readerGuide}[1]{%
  \noindent\textbf{読むときのポイント}\par
  \begin{itemize}\setlength{\itemsep}{2pt}
    #1
  \end{itemize}\vspace{2mm}
}

\newcommand{\chaptertakeaway}[1]{%
  \par\medskip
  \noindent\textbf{まとめ:} #1\par
}

% =========================================================
% Notation (example)
% =========================================================
\newcommand{\call}{\mathcal{C}_{\mathrm{all}}}
\newcommand{\lone}{\mathcal{L}_{\mathrm{one}}}

% =========================================================
% Algorithm captions in Japanese
% =========================================================
\makeatletter
\renewcommand{\ALG@name}{アルゴリズム}
\makeatother
\algtext*{EndWhile}
\algrenewcommand\algorithmicrequire{\textbf{Input:}}
\algrenewcommand\algorithmicensure{\textbf{Output:}}

% =========================================================
% Theorem environments (読みやすい順に)
% =========================================================
\theoremstyle{definition}
\newtheorem{dfn}{定義}
\newtheorem{thm}{定理}
\newtheorem{lem}{補題}
\newtheorem{cor}{系}
\renewcommand{\proofname}{\bf{証明}}

% =========================================================
% Page style for bachelor thesis
% =========================================================
\pagestyle{bachelorthesis} % 卒論はこっち
% \pagestyle{masterthesis} % 修論はこっち

% =========================================================
% Meta
% =========================================================
\title{ユークリッド辺重みをもつ\\単位円盤グラフにおける\\最小全域木問題の\\分散近似アルゴリズム}
\author{木村 涼}
\supervisor{泉 泰介\\北村 直暉}
\deadline{令和8年2月9日}

\begin{document}

\titlepage

% =========================================================
% Abstract
% =========================================================
\abstract{
  \textbf{背景:} 〜〜が重要である。\\
  \textbf{目的:} 本稿では〜〜を目指す。\\
  \textbf{手法:} 〜〜を用いて〜〜する。\\
  \textbf{結果:} 〜〜を達成した(/見積もった)。\\
  \textbf{貢献:} 本研究の貢献は(1)〜〜,(2)〜〜である。
}

\keyword{
  キーワード1,キーワード2,キーワード3
}

\toc
\newpage

% =========================================================
\section{はじめに}
% =========================================================
\chapterintro{研究の「背景→課題→本稿の貢献」を短くつなぎ、読み手が全体像を掴めるようにします。}

\readerGuide{
  \item まず\textbf{何が問題で、なぜ重要か}(背景と課題)
  \item 次に\textbf{何をどこまで解くか}(設定と到達点)
  \item 最後に\textbf{何が新しいか}(貢献)
}

\subsection{背景}
本文を書いてください。導入では「専門用語を最小限」にして、初見でも読めるようにします。
引用する場合はこのように書くことが知られている\cite{hoge}。並べて書いたってかまわない\cite{fuga,puyo}。

\subsection{本稿の貢献}
\begin{itemize}\setlength{\itemsep}{2pt}
  \item (貢献1)〜〜を提案する。
  \item (貢献2)〜〜を理論的に示す。
  \item (貢献3)〜〜を実装・評価する(該当する場合)。
\end{itemize}

\subsection{論文の構成}
\begin{itemize}\setlength{\itemsep}{2pt}
  \item 第2章:関連研究
  \item 第3章:準備(記法・定義)
  \item 第4章:提案手法
  \item 第5章:評価
  \item 第6章:まとめ
\end{itemize}

\chaptertakeaway{この章を読み終えると、「何を解く論文で、どこが新しいか」が説明できる状態になります。}

% =========================================================
\section{関連研究}
% =========================================================
\chapterintro{先行研究を「何ができて、どこが足りないか」に整理し、本稿の立ち位置を明確にします。}

\readerGuide{
  \item 似た問題設定の手法を\textbf{目的・仮定・計算量}で比較する
  \item 本稿が埋めるギャップを\textbf{一言で言える}ようにする
}


\chaptertakeaway{「先行研究A/Bに対して本稿はここが違う」と言える状態になります。}

% =========================================================
\section{諸定義}
% =========================================================
\subsection{記法の定義}
\begin{dfn}[グラフの記法の定義]
    $G=(V,E)$を無向単純連結グラフとする.頂点数を$n=|V|$,辺数を$m=|E|$とする.任意の辺$e=(u,v)\in E$に対し,その重みを$w(u,v)$とする.
\end{dfn}
\begin{dfn}[幾何学的記法の定義]
    各頂点$v\in V$はユークリッド平面上の座標$(x_v,y_v)\in \mathbb{R}^2$を持つ.2頂点$u,v$間のユークリッド距離を\[d(u,v)=\sqrt{(x_u-x_v)^2+(y_u-y_v)^2}\]と定義する.
\end{dfn}

\subsection{用語の定義}
\begin{dfn}[CONGESTモデル]
    CONGESTモデルは同期型の分散計算モデルである.計算はラウンド単位で進行し,各ラウンドにおいて各ノードは隣接ノードに対し$O(\log n)$ビット長のメッセージを双方向に伝送できる.各ノードは無限の局所計算能力を持つと仮定する.また,各ノードは$O(\log n)$長の自然数値によるIDが付与されており,自身の隣接ノードすべてのIDは既知であるとする.各ノードはグラフのトポロジに関する知識を持たない.
\end{dfn}
\begin{dfn}[ユークリッド辺重みをもつ単位円盤グラフ]
    平面上の頂点集合$V\subset \mathbb{R}^2$に対し,任意の$u,v\in V$について,ユークリッド距離$d(u,v)\le 1$のとき辺$(u,v)$を張ったグラフを単位円盤グラフと呼ぶ.特に,各辺の重みをユークリッド距離$w(u,v)=d(u,v)$で与えたものをユークリッド辺重みをもつ単位円盤グラフと呼ぶ.
\end{dfn}
\begin{dfn}[最小全域木(MST)]
    重み付き無向グラフ$G=(V,E,w)$において,全頂点を含む木のうち,辺重みの総和が最小となるものを最小全域木(Minimum Spanning Tree; MST)という.
\end{dfn}
\begin{dfn}[$\alpha$-近似]
    最小化問題において,アルゴリズムが$\alpha$-近似であるとは,最適解$S^*$に対して求めた解$S$が$S^*\le S\le \alpha \cdot S^*$を満たすことをいう.
\end{dfn}
%\chapterintro{以降で使う記法・定義をそろえます。読者が途中で迷わないための章です。}

% \subsection{記法}
% 本文を書いてください。

% \begin{dfn}[辺集合]
%   グラフ$G$の辺集合を$E(G)$とする。
% \end{dfn}
% % 図・表は「何を見るか」を先に書くと親切
% \subsection{図と表の例}
% \noindent\textbf{見てほしい点:} 図\ref{fig:a}は「配置」、表\ref{tab:placeholder}は「数値比較」を示す例です。

% \begin{figure}[htbp]
%   \centering
%   \includegraphics[width=0.7\linewidth]{figure/a.eps}
%   \caption{図のキャプション(a.eps)}
%   \label{fig:a}
% \end{figure}

% \begin{figure}[htbp]
%   \centering
%   \includegraphics[width=0.7\linewidth]{figure/b.png}
%   \caption{図のキャプション(b.png)}
%   \label{fig:b}
% \end{figure}

% \begin{table}[htbp]
%   \centering
%   \caption{表のキャプション(何を比べる表かを一文で)}
%   \label{tab:placeholder}
%   \begin{tabular}{l|c|r}
%     項目 & 値1 & 値2 \\
%     \hline
%     A  & 1  & 10 \\
%     B  & 2  & 20 \\
%   \end{tabular}
% \end{table}

% \chaptertakeaway{ここまでで、以降の章を読むための「辞書」を揃えました。}

% =========================================================
\section{提案アルゴリズム}
% =========================================================
\subsection{概要}
\subsection{具体的な動作}
本節では,提案アルゴリズムの具体的な動作について説明する.
\begin{algorithm}[H]
    \caption{分散定数近似MST構築アルゴリズム}
    \begin{algorithmic}[1]
        \Require ユークリッド辺重みをもつ単位円盤グラフ$G=(V,E)$
        \Ensure$G$上のMST
        \State 独立点集合$I$を求める
        \State $v\in I$をリーダーノードとする
        \State $v$から距離4以下の独立点集合$B_4(v)=\{u \mid d_G(u,v) \le 4\}$を求める
        \State $G[B_4(v)]$上でMST$T_v$を構築する
        \While{$V_(T_v)\subset V$}
            \State $T_v$に新しく追加された独立点に対し,距離9の独立点集合$I'$を求める
            \For{$u\in I'$}
                \State $B_4(u)$に誘導される部分グラフ上でMST$T_u$を構築する
            \EndFor
            \For{構築された部分木$T$}
                \State 独立点を根とする根付き木に変換する
                \For{$u\in T|u\notin T_v$}
                    \State 親方向の辺を$T_v$に追加する
                \EndFor
            \EndFor
        \EndWhile
    \end{algorithmic}
\end{algorithm}
本アルゴリズムでは,まず独立点集合を求め,その中からリーダーノードとなる頂点$v$をび,それから半径4以内の局所領域においてMSTを構築する.
次に,新たに追加された独立点の距離が9であるような独立点集合を求める.これにより,各局所計算は互いに干渉せず,分散的に実行可能となる.
各局所MSTは,独立点を根とする根付き木へと変換され,未接続の頂点は親方向への辺を追加することで全体のMSTが拡張される.
この操作を反復することで,最終的に全頂点を含むMSTが得られる.

% \chapterintro{手法の全体像→詳細→正しさ(または理屈)の順に説明します。}

% この章では以下の定理を証明する。
% \begin{thm}
%   $P=NP$となるのは、$P=0$または、$N=1$のときに限る。
% \end{thm}

% \subsection{全体像}
% 本文を書いてください。ここでは「何を入力として受け取り、何を出力するか」を先に書くと読みやすいです。

% \subsection{補題と証明の例}
% \begin{lem}
%   任意のグラフ$G$に対して,$E(G)$は有限である。
% \end{lem}
% \begin{proof}
%   $G$は有限個の頂点からなると仮定する。各頂点は有限本の辺のみを持つため,
%   辺集合$E(G)$も有限である。
% \end{proof}

% \subsection{アルゴリズムの例}
% \begin{algorithm}[H]
%   \caption{アルゴリズム名(何をするか)}
%   \begin{algorithmic}[1]
%     \Require 入力の説明
%     \Ensure 出力の説明
%     \State 初期化を行う
%     \While{条件}
%     \State 処理を行う
%     \EndWhile
%   \end{algorithmic}
% \end{algorithm}

% \chaptertakeaway{この章の最後で、手法を「口頭で説明」できるのが理想です。}

% =========================================================
\section{評価(実験系のテーマじゃなければ省略可)}
% =========================================================
\chapterintro{提案手法が「どれくらい良いか」を、比較と再現性の観点で示します。}

\readerGuide{
  \item 何と比べるか(\textbf{ベースライン})を最初に宣言
  \item 設定(データ/環境)を\textbf{再現できる粒度}で書く
  \item 主要結果は\textbf{1図1メッセージ}で示す
}

\chaptertakeaway{結果だけ見ても「提案の良さ」が伝わるのがゴールです。}

% =========================================================
\section{まとめ}
% =========================================================
\chapterintro{本稿で得られた結論と、次にやるべき課題を短く整理します。}

\subsection{まとめ}
\begin{itemize}\setlength{\itemsep}{2pt}
  \item 本稿では〜〜を行った。
  \item その結果、〜〜が分かった/達成した。
\end{itemize}

\subsection{今後の課題}
\begin{itemize}\setlength{\itemsep}{2pt}
  \item 課題1:〜〜
  \item 課題2:〜〜
\end{itemize}

\bibliographystyle{unsrt}
\bibliography{biblio}

\end{document}
